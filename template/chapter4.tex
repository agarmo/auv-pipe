\chapter{Discussion}
	This chapter will summarize and discuss the results in the previous chapter.

\section{Guidance System}
	The guidacen system are greately simplifyed. In this section these simplifications will be adressed
	and other possible errors and shorcomings will be treated. 

	This Guidance system are designed for long, straight pipeline streches. It will not be able to handle
	sudden turns in the pipeline direction, without spcifying this in the guidance system. This can be
	included as waypoints where the pipeline are changing, and be included as a condition; \textit{when the AUV
	reaches certain position the direction of the pipeline changes}. This requiers very exact knowledge
	about the pipeline and are rearly the case. TThis is why there should be a more autonomous way of
	doing this. This motivates the use of more sensros than just a camera to follow the pipeline. The
	camera have a very limited field of view, usually restricted to less than 3 meters. A Side Scan Sonar
	comined wiht a Forward Looking Sonar, which provides sensor data of the pipeline in front of the AUV
	will give the guidance system some data to descide and predict what it will do if there is a sharp
	turn in the pipeline. 

	Sharp turns are rare, and are a product of T-junctions and other couplings of pipelines. These
	junctions are usually well documented, and given good and exact locations. But if the navigation
	sensors of the AUV have large uncertanties, and from the AUV it might look as they are in the wrong
	place. This suggests that information about the pipeline should be treated with care. Because the
	errors in an AUV navigation system might be substantial and provide that \'a priori information about
	the pipeline will be unusable. 

	The navigation system of \textit{HUGIN 1000} are a Velocity Aided Inertial Navigation System. This
	utilises a Doppler Velocity Log to measure the velocity relatively to the sea bottom and input this to
	the INS system. The INS systems usually installedn on the \textit{HUGIN} are in the 1 nmi/h class, i.e
	the INS system drifts less than 1 nmi/h. This results in a drift in the navigation system equal to
	0.11 \% of the traveled distance along the track, and about 0.03 \% error in the across distance of a
	stright line track. \cite{INS_Hugin}. This can be enough to throw the guidance system off course,
	because the field of view of the camera are relatively small.

	***********************************************

	There are ways of improving the INS drift, one is to use GPS update fixes, but this requiers to
	surface the AUV once in a while. This is of course not a good idea when the AUV are at great depths.
	There are posibilities to use sea bottom anchored position bouys, which exact position are known and
	the AUV might use these bouys by pinging them and getting a updated position estimate. This is a good
	idea if the pipeline infrastructure admints this. Say that this position bouys are placed at the same
	time as the pipeline are layed, but this is a costly affair. 

	Too use a Ultar Short Base Line (USBL) are another posibility. A USBL transducer are mounted on a
	ship, which has a GPS location fix. The AUV then pings the USBL transducer regularily and the position
	are determined exactly. This requiers a ship stationed in the area where the AUV are carrying out the
	mision. The autonomicity of the AUV are the reduced, and the vehicle are not capable of operating on
	its own. 

	**********************************************

	The problem regarding when $\psi \rightarrow 2\pi$ problem, there are a number of solutions for this.
	The first is to limit the sensor output, which are the case in the real world, since a compass
	measuring \textit{yaw} only are defined for $(0, 2 \pi)$. The controller can handle this by including
	a chech wheter if its heading are larger than $\pi$, the given command will be to the right, and
	opposite if the measured heading are smaller than $\pi$.






\section{Roll Stabilization}
	Roll stabilization due to downward looking camera. This is probably not needed because the roll angle
	never reaches more than a couple of degrees in magnitude at the probable velocities of the vehicle. 


\section{Energy Consumption}


\section{Optimal Search Pattern}




