\chapter{Conclusions}
	In this report the topic of designing a guidance system capable of pipeline inspeciton 
	for the \hugin AUV have been treated. This report have looked at some basics regarding pipeline
	inspection. A literature study on how pipeline inspection can be preformed was done, and summarised in
	the report. Some basics about guidance was reviewed.

	A proposed guidance system were also designed. The system includes a flight-mode controller, a
	guidance algorithm and a Kalman filter to fuse measurements and prior knowledge about the pipeline
	together. The filter will also smooth the output from the sensors to provide a continous reference for
	the guidance algorithm. 

	The controller designed are three independent PI-, and PID-controllers. Which were tuned according to
	stability and simulations.

	The behaviour of the guidance system were defined and examined. Some search patterns were discussed,
	and found that they should be customised with regard to how well the mission area are known and how
	certain the data about the pipeline are. 
	
	The system were implemented in Matlab/simulink and 4 independent simulations were run, demonstrating
	different abilities with the given guidance system. The system shown results and the given setup
	worked, also in the presence of ocean current. The simulations were discussed with regard to energy
	efficiency and pipeline following capabilities. The most optimal path to follow concerning energy
	optimality when the velocity are constant is the shortest path, in most cases the linear segments
	between two points. 
	
	Whenever the AUV are tracking, the the motion above the pipeline should be constrained to be in the
	vicinity of the pipeline but not strictly above it all the time. The primary mission of the inspection
	should be to provide good, easy-to-analyse sensor data which in most cases include trying to maneuver 
	the AUV as smoothely as possible. 

	The camera should be aided with other sensors, such as Side Scan Sonar and Multibeam Echosounders.
	Theses sensors help with detection of the pipeline and will also provide data about the condition of
	the pipeline. 

	The strict modularity of the guidance system presented in this report allows for easy upgrades and
	improvment to the different blocks in the system. In further work with this problem the guidance
	system will be easy to upgrade, with more advanced controllers or guidance algorithms. The simulation
	environment are present and gives good results, but the control of the system should be better in
	order to implement in a real application.

	
\section{Further Work}
	As stated before the whole pipeline inspection problem are not treated in a single report. There are
	much work do be done before this can be transformed into real life application.
	
	First the guidance should handel three dimentional guidance with nonlinear paths. A possible way to 
	do this is to use unified guidance controller proposed in \cite{control-concept-AUV}. This is a 
	controller for the whole non-zero speed regieme.

	The Kalman filter should be designed to include readings from other sensors like Side Scan Sonar,
	Echosounders, and other possible sensors available for the inspection mission. The filter should be
	expanded to be capable of three dimentional prediction and a more sofisticated prediction might be
	used to get good results for the predictions. 

	

	\begin{itemize}
		\item Optimal guidance and path planning
	\end{itemize}


