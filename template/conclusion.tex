\chapter{Conclusion}
	In this report, the topic of designing a guidance system capable of pipeline inspection 
	for the \hugin AUV has been treated. This report have looked at some basics regarding pipeline
	inspection. A literature study on how pipeline inspection can be preformed was done, and summarised in
	the report. Some basics about guidance was reviewed.

	A proposed guidance system was also designed. The system includes a flight-mode controller, a
	guidance algorithm and a Kalman filter to fuse measurements and prior knowledge about the pipeline
	together. The filter will also smooth the output from the sensors to provide a continuous reference
	for the guidance algorithm. 

	The controller designed consists of three independent PI-, and PID-controllers, which were tuned according to
	stability and simulations. They are tuned with regard to velocities around $1$ m/s. Especially the heading
	controller needs to be re-tuned to make it compatible with velocities higher than this.

	The behaviour of the guidance system was defined and examined. Some search patterns were discussed,
	and found that they should be customised with regard to how well the mission area are known and how
	certain the data about the pipeline are. 
	
	The system was implemented in Matlab/simulink and 4 independent simulations were run, demonstrating
	different abilities with the given guidance system. The system gave results and the given setup
	worked, also in the presence of ocean current, but are not as robust as it should be. The simulations
	were discussed with regard to energy
	efficiency and pipeline following capabilities. The most optimal path concerning energy
	optimality when the velocity are constant, is the shortest path, in most cases the linear segments
	between two points. 
	
	Whenever the AUV is tracking, the motion above the pipeline should be constrained to be in the
	vicinity of the pipeline, but not strictly above it all the time. The primary mission of the inspection
	should be to provide good, easy-to-analyse sensor data, which in most cases include trying to maneuver 
	the AUV as smoothly as possible. 

	The camera should be aided with other sensors, such as Side Scan Sonar and Multibeam Echosounders.
	Theses sensors help with detection of the pipeline and will also provide data about the condition of
	the pipeline. 

	The strict modularity of the guidance system presented in this report allows for easy upgrades and
	improvement to the different blocks in the system. In further work with this problem the guidance
	system will be easy to upgrade, with more advanced controllers or guidance algorithms. The simulation
	environment present, gives good results.

	
\section{Further Work}
	There are much work do be done before this can be transformed into real life application.
	
	First the guidance system should handle three dimensional guidance with nonlinear paths. A possible way to 
	do this is to use unified guidance controller proposed in \cite{control-concept-AUV}. This is a 
	controller for the whole non-zero speed regime.

	The Kalman filter should be designed to include readings from other sensors like Sidescan Sonar,
	Echosounders, and other possible sensors available for the inspection mission, such as a Synthetic
	Aperture Sonar. The filter should be
	expanded to be capable of three dimensional prediction and a more sophisticated prediction might be
	used to get good results for the predictions. 
	
	It should be room for making the guidance system control the inspection velocity. If there are large 
	stretches of pipeline
	which requires little maneuvering, the speed can be increased or if there the sensors show parts of the
	pipeline which shows signs of corrosion, the inspection velocity can be slowed to give more data about
	the area.

	Develop some kind of a \textit{power} mode and \textit{economy} mode for the guidance system, where 
	the \textit{power} mode can be used when there are strong currents in the area, to help the AUV
	continue the tracking procedure, or to quickly get the AUV to the desired location. The
	\textit{economy} mode can be utilised when tracking the pipeline and environmental forces are not very
	dominating. 

	
