\chapter{Introduction}
	Today there are over 3.5 million km of transmission pipelines in the world. 231 900 km of pipelines
	are planned or under construction \cite{DNV_pipelines}. The pipelines currently in place need to
	continue operating, many of them far longer than they were initially intended. Great effort are taken
	to inspect and maintain the pipelines to keep them in satisfying order.  The pipeline needs to be 
	in working order to be operational.
	Potential leaks might cause irreversible environmental damages, and companies looses money when the
	flow of oil or gas are stopped. 

	There are two ways of carrying out pipeline inspections; \textit{internal} and \textit{external} 
	inspection. The internal inspection methods, includes stopping the flow of whatever is in the pipeline,
	open it an insert a Pipeline Inspection Gauge (PIG) which travels the inside the pipeline and uses various
	sensors to determine the state of the pipeline. The other method, \textit{exterior} pipeline
	inspections are today mostly done using Remotely Operated Vehicles, ROVs. The ROVs are untethered
	unmanned underwater vehicles, and are the work horse of the offshore industry. They are versatile
	tools capable of accomplishing most missions associated with pipeline inspections and repair. However,
	they need well-equipped, expensive support vessels and a large crew to accompany the inspection
	mission. The fact that this is a tethered vessel, it has a very limited operating radius. 

	An Autonomous Underwater Vehicle (AUV) is a suiting tool for pipeline inspection. An AUV is a
	untethered unmanned underwater vehicle. For the AUV to be able to perform an inspection mission it
	should be able to navigate and make decisions for itself. Opposed to the ROVs the operation radius of
	the AUV are limited by power and battery life. An AUV can be small enough to be launched form, shore
	or small ships. It usually does not require the large support crew needed for an ROV mission.

	The speed of the inspections might also be of importance. A typical ROV has cruise speeds from $1-2$
	knots ($0.5-1$ m/s), while an AUV has cruise speeds in the regime $2-6$ knots ($1-3$ m/s). This
	inclines that an AUV might cover larger area of pipelines than an ROV. 

	British Petroleum estimates that they can save up to 30 \% by using AUVs for pipeline inspections
	instead of ROVs.\cite{PhD_lecture}

	The technology needed for such a mission are present, and there are a number of companies developing
	AUVs for pipeline following. SeaByte and Subsea7 have conducted a successfull pipeline inspection
	mission using the \textit{Geosub} AUV. They claimed the world record in the longest uninterrupted
	pipeline inspection mission. The AUV inspected 22.2 km of pipeline at $4$ knots without being
	interrupted. \cite{Seabyte}
	
	This report will look at the possibility to give the Kongsberg Maritime \textit{HUGIN 1000} AUV the
	given abilities to track and follow a subsea pipeline. The AUV is controllable in 5 degrees 
	of freedom (DOF) and assumed stable in the \textit{roll} degree of freedom, which gives it
	hovering capabilities, although this will not be used when designing the guidance system.

	The pipeline detection equipment is a downward looking camera with sufficient lighting to operate at about
	3-5 meters above the seabottom. Of course the visibility conditions will change according to depth and
	amount of particles in the water(naval snow). 

	This report considers how to automate the pipeline inspections process. It will consider the
	possibility that the pipeline is buried under mud and not visible to the camera or buried on purpose
	to make it more robust towards environment forces and erosion, in both cases the important thing 
	is to reacquire the pipeline on the other side of the buried stretch. In that case some kind of an
	estimator will be used to predict where the pipeline is headed.
	
	A pipeline inspection mission can be divided into 4 parts:
	\begin{enumerate}
	 \item Initialisation and initial descent
	 \item Search for and Acquire pipeline
	 \item Track and Inspect pipeline
	 \item Ascent to the surface and deliver the acquired data
	\end{enumerate}
	
	The \textit{Search and Acquire} part will need the vessel to move in some kind of search pattern if
	the pipeline is not located exactly where the initial position data states it to be. This search
	pattern should also be used if the pipeline is lost during tracking. 
	
	The \textit{Track and Inspect} part needs a guidance system which is capable of keeping the vessel 
	on top of the pipeline independent of the current in the area and other possible disturbances. This is
	necessary because the primary mission for the AUV is to provide video of the pipeline which can be
	used to determine the state and well-being of the pipeline.
	
	This report is divided into four chapters;
	\begin{enumerate}
	 \item Theory. Describes the necessary theory needed to understand the problem, and contains a summary
	of literature on the pipeline following subject.
	 \item Modeling, assumptions about the model and how it is modeled.
	 \item The simulation and implementation of the model.
	 \item Discussion of the results given by the simulations.
	\end{enumerate}

	Last, this report will consider the simulation, and document on the results given by the implemented 
	guidance system. There will be a discussion on how well the guidance system preformed and how accurate
	the data from the simulation is. 
	
	
	

